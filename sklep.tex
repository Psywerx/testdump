\chapter{Sklepne ugotovitve}


% zaključek (odstavki: kaj smo naredili in kje je koda, kaj nam to omogoča, 
% kaj bi lahko še naredili)

Z izdelavo dodatka za program Orange smo zakljucili delo na diplomski nalogi.
Koda izdelanega dodatka se nahaja na git ....


% S tem dodatkom smo omogočili dostop do podatkov Svetovne banke tudi
% uporabnikom programa Orange brez znanja o samem programskem vmesniku SB.
% 
% S tem dodatkom smo olajsali dostop do velike zbirke podatkov Svetovne banke,
% ki jih lahko sedaj znotraj programa Orange za svoje delo uporabi kdorkoli.
%
%- lazje vzdrzevanje in posodabljanje kode
%
%
%
%Tako smo omogočili lazje


Nas grafični dodatek za dostop do podatkov indikatorjev lahko nadgradimo tako,
da uporabnikom grafičnega vmesnika omogočimo večjo izbiro oblik izhodnih
podatkov in natančnejse presejanje rezultatov. Dodamo lahko tudi več
metapodatkov na posamezne stolpce Orange tabele, ki nam omogočijo boljšo
predstavnost v ostalih Orange gradnikih. V grafični vmesnik za dostop do
podnebnih podatkov lahko dodamo še možnost izbire vodotočnih območji meritev.
Za boljšo predstavo bi lahko postopek izbire drzav, regij in vodotočnih
območij (Slika \ref{climate_data_api_basins}) omogočili prek interaktivnega zemljevida sveta.


- dodamo metapodatke tudi climate gradniku
- boljsa pokritost testov


- V data sets skupino bi lahko dodali se gradnik za katerega od drugih v uvodu
nastetih spletnih programskih vmesnikov
