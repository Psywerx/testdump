\chapter{Sklepne ugotovitve}


% zaključek (odstavki: kaj smo naredili in kje je koda, kaj nam to omogoča, 
% kaj bi lahko še naredili)

Z izdelavo dodatka za program Orange smo zakljucili delo na diplomski nalogi.
Koda izdelanega dodatka se nahaja na git ....


% S tem dodatkom smo omogo"cili dostop do podatkov Svetovne banke tudi
% uporabnikom programa Orange brez znanja o samem programskem vmesniku SB.
% 
% S tem dodatkom smo olajsali dostop do velike zbirke podatkov Svetovne banke,
% ki jih lahko sedaj znotraj programa Orange za svoje delo uporabi kdorkoli.
%
%- lazje vzdrzevanje in posodabljanje kode
%
%
%
%Tako smo omogo"cili lazje


Nas grafi"cni dodatek za dostop do podatkov indikatorjev lahko nadgradimo tako,
da uporabnikom grafi"cnega vmesnika omogo"cimo ve"cjo izbiro oblik izhodnih
podatkov in natan"cnejse presejanje rezultatov. Dodamo lahko tudi ve"c
metapodatkov na posamezne stolpce Orange tabele, ki nam omogo"cijo bolj"so
predstavnost v ostalih Orange gradnikih. V grafi"cni vmesnik za dostop do
podnebnih podatkov lahko dodamo "se mo"znost izbire vodoto"cnih obmo"cji meritev.
Za bolj"so predstavo bi lahko postopek izbire drzav, regij in vodoto"cnih
obmo"cij omogo"cili prek interaktivnega zemljevida sveta.


- dodamo metapodatke tudi climate gradniku
- boljsa pokritost testov
