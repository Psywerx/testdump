\chapter{Spletni viri indikatorjev dr"zav sveta}

% Na spletu je mnogo prosto dostopnih virov oz. baz podatkov. Ti imajo programske
% vmesnike, ki omogo"cajo dostop do raznovrstnih podatkov, kot so npr. seznami 
% stopnje ogro"zenosti zivali po dr"zavah 
% \fnurl{http://apiv3.iucnredlist.org/api/v3/docs},
% Nasini podatki meritev in slike vesolja
% \fnurl{https://api.nasa.gov/}
% seznam knjig z ocenami in povezavami med uporabniki
% \fnurl{https://www.goodreads.com/api},
% zgodovina meteoroloskih meritev
% \footnote{\label{climate_api}\url{http://climatedataapi.worldbank.org/}},
% razni indikatorji stopnje razvoja dr"zav
% \footnote{\label{indicator_api}\url{http://api.worldbank.org/}}.
% 


Pri nalogi smo se osredoto"cili na dva programska vmesnika za dostop podatkov 
Svetovne banke, to sta zgodovina meteorolo"skih meritev 
\textsuperscript{\ref{climate_api}} in 
razni indikatorji stopnje razvoja dr"zav \textsuperscript{\ref{indicator_api}}.
Za uporabo podatkovne baze Svetovne banke smo se odlo"cili, ker zdru"zuje in na
enovit na"cin predstavi podatke iz ve"c razli"cnih virov. Podatkovni viri za 
indikatorje
stopnje razvoja dr"zav so:

% 1 IUCN 2016. IUCN Red List of Threatened Species. Version 2016-1 <www.iucnredlist.org>
% http://apiv3.iucnredlist.org/api/v3/docs
% 
% 2 https://api.nasa.gov/#getting-started
% 
% 3 https://www.goodreads.com/api
% 
% 4 http://climatedataapi.worldbank.org/
% 
% 5 http://api.worldbank.org/



\begin{itemize}  
\item World Development Indicators, 
\item Global Development Finance, 
\item African development Indicators, 
\item Doing Business,
\item Enterprise Surveys, 
\item Millennium Development Goals, 
\item Education Statistics, 
\item Gender Statistics,
\item Health and Nutrition Statistics, 
\item IDA Results Measurement System.
\end{itemize}  

Podatkovni viri za klimatske meritve so pridobljeni s svetovnih meteorolo"skih 
postaj.


Svetovna banka omogo"ca dostop do podatkov prek vmesnika --slovensko-- 
% TODO
% kako predstavimo REST XML in JSON
(ang. Representational state transfer, REST), ki ponuja 
veliko mo"znosti za iskanje in presejanje rezultatov. Pri vsaki REST poizvedbi
lahko dolo"cimo "zeljeno obliko odgovora. Za poizvedbe o informacijah 
indikatorjev sta na voljo obliki --slovensko-- XML in --slovensko-- JSON. 
Programski vmesnik meteorolo"skih
meritev pa ponuja samo obliko JSON. Za konsistentnost in la"zjo berljivost smo
na obeh vmesnikih uporabili obliko JSON. 


\section{Podatki indikatorjev razvoja dr"zav}



Programski vmesnik indikatorjev razvoja dr"zav Svetovne banke omogo"ca dostop
do podatkov preko 16.000 raznih indikatorjev. Podatki indikatorjev so merjeni
od leta 1960 dalje v mese"cnem, "cetrtletnem ali letnem intervalu. Poleg podatkov
indikatorjev nam ta programski vmesnik omogo"ca tudi dostop do ve"cine
metapodatkov s katerimi lahko presejamo in natan"cneje dolo"cimo na"so poizvedbo.
Seznami metapodatkov so:
\begin{itemize}
\item viri podatkov in njihovi opisi (Catalog Source Queries
	\fnurl{http://api.worldbank.org/sources}),
\item seznam dr"zav, skupin dr"zav in regij z identifikatorji (ang. :w
  Country Queries
	\footnote{\label{country_list}\url{http://api.worldbank.org/countries}}),
\item razdelitev vi"sin dohodkov z identifikatorji (Income Level Queries
	\fnurl{http://api.worldbank.org/incomeLevels}),
\item seznam vseh indikatorjev (Indicator Queries
    \footnote{\label{indicators_list}\url{http://api.worldbank.org/indicators}}),
\item seznam tipov posojil (Lending Type Queries
	\fnurl{http://api.worldbank.org/lendingTypes}),
\item seznam tem (Topics \fnurl{http://api.worldbank.org/topics}).
\end{itemize}


Za dostop do podatkov posameznega indikatorja, potrebujemo identifikator
indikatorja s seznama vseh 
indikatorjev\textsuperscript{\ref{indicators_list}} in kodo ene ali ve"c
dr"zav oziroma regij s seznama 
dr"zav\textsuperscript{\ref{country_list}}. Privzeta vrednost za
koli"cino podatkov na stran je 50, zgornja meja pa ni strogo dolo"cena, vendar
je odvisna od velikosti odgovora. Ugotovili smo tudi, da se zanesljivost
programskega vmesnika manj"sa z ve"cjo koli"cino podatkov na stran. V na"sem 
programu smo se omejili na 1000 podatkov na stran, kar se je izkazalo za 
uporabno razmerje med hitrostjo in zanesljivostjo prenosa. Vsi seznami in 
metapodatki, ki so na voljo s programskim vmesnikom indikatorjev razvoja imajo
enako osnovno obliko (Primer \ref{basic_response}).


\begin{snippet}
\begin{center}
\begin{lstlisting}
[
    {
    	"page": 4,
		"pages": 137,
		"per_page": "50",
		"total": 6831
    },
    [
        {<podatki>},
        ...
    ]
]
\end{lstlisting}
\end{center}
% TODO : dolgi linki
% http://api.worldbank.org/en/countries/all/indicators/16.1\_TRANS.ENERGY.INTENSITY?format=json\&page=4
\caption{Osnovna oblika odgovora programskega vmesnika Svetovne banke, ob 
veljavni poizvedbi. Prvi element opisuje koli"cino dobljenih in "stevilo vseh 
podatkov, drugi element pa vsebuje s stranjo in "stevilom podatkov na stran 
dolo"ceni izsek celotnih podatkov.}
\label{basic_response}
\end{snippet} 



\subsection{Opis seznama indikatorjev}

Programski vmesnik Svetovne banke za indikatorje razvoja nam ponuja seznam 
vseh indikatorjev\textsuperscript{\ref{indicators_list}} z imeni, opisi, 
identifikatorji in drugimi metapodatki (Primer \ref{indicator_response}).
Programski vmesnik nam tudi omogo"ca presejanje podatkov glede na vir podatkov
indikatorja (ang. source).

\begin{snippet}
\begin{center}
\begin{lstlisting}
{
    "id": "1.0.HCount.2.5usd",
    "name": "Poverty Headcount (\$2.50 a day)",
    "source": {
        "id": "37",
        "value": "LAC Equity Lab"
    },
    "sourceNote": "The poverty headcount index measures 
                   the proportion of the population
                   with daily per capita income (in 
                   2005 PPP) below the poverty line.",
    "sourceOrganization": "LAC Equity Lab tabulations
                           of SEDLAC (CEDLAS and the
                           World Bank).",
    "topics": [
        {
            "id": "11",
            "value": "Poverty "
        }
    ]
}
\end{lstlisting}
\end{center}
\caption{Podatki indikatorja 
% TODO
% Ali moram ime indikatorja oznaciti (naprimer velika zacetnica)
% > Podatki indikatorja Stopnja rev...
stopnja rev"s"cine pri dohodku 2,5 dolarja na dan.}
\label{indicator_response}
\end{snippet} 

\subsection{Opis seznama dr"zav}

Seznam dr"zav\textsuperscript{\ref{country_list}}  na programskem vmesniku 
Svetovne banke vsebuje podatke o imenih, opisih, ISO-3166-1 alpha kodah, 
regijah in druge metapodatke (Primer \ref{country_response}). Programski
vmesnik nam tudi omogo"ca presejanje seznama dr"zav po naslednjih poljih:
\begin{description}
\item [id] koda,
\item [region] regija,
\item [incomeLevel] vi"sina dohodka,
\item [lendingType] tipov posojil. % TODO preveri prevod!
\end{description}

Ta seznam ne vsebuje zgolj samo dr"zav, ampak tudi regije in skupine dr"zav, 
zdru"zenih glede na razli"cne kriterije (vi"sine dohodka, velikost, stopnja
razvoja). Poleg tega zgornji seznam vsebuje tudi nekatere izjeme kot je trenutno
Kosovo. V nadaljevanju bomo za vse na"stete tipe lokacijskih podatkov
uporabljali besedo ``dr"zave''.

\begin{snippet}
\begin{center}
\begin{lstlisting}
{
    "id": "ABW",
    "iso2Code": "AW",
    "name": "Aruba",
    "region": {
        "id": "LCN",
        "value": "Latin America & Caribbean "
    },
    "adminregion": {
        "id": "",
        "value": ""
    },
    "incomeLevel": {
        "id": "HIC",
        "value": "High income"
    },
    "lendingType": {
        "id": "LNX",
        "value": "Not classified"
    },
    "capitalCity": "Oranjestad",
    "longitude": "-70.0167",
    "latitude": "12.5167"
},
\end{lstlisting}
\end{center}
\caption[some]{Izsek podatkov veljavne poizvedbe dr"zav.}
\label{country_response}
\end{snippet} 


\subsection{Dostop do podatkov indikatorjev}


Za pridobivanje podatkov indikatorjev se uporablja dostopna tocka \\
% TODO: kako z dolgimi linki in to spravit v kodo
http://api.worldbank.org/countries/<country\_list>/indicators/<indicator\_id>, \\
kjer je:


\begin{description}
\item [country\_list] s podpi"cjem lo"cen seznam kod izbranih dr"zav, ki jih 
	  preberemo iz polja ``id'' ali ``iso2Code'' \ref{country_response}, ali pa 
      klju"cna beseda ``all'',
\item [indicator\_id] polje ``id'' indikatorja \ref{idicators_response} s seznama
      indikatorjev.
\end{description}

API omogo"ca "ze z eno samo poizvedbo dostop do podatkov ene dr"zave, ve"c izbranih 
dr"zav hkrati ali pa do podatkov vseh dr"zav dostopamo s klju"cno besedo ``all''.
Slabost API-ja SB je v tem, da ne moremo z eno poizvedbo dostopati do podatkov
ve"c indikatorjev hkrati. Podatke lahko presejamo po naslednjih poljih:
\begin{description}  
\item [MRV] "stevilo zadnjih meritev,
\item [frequency] pogostost vzor"cenja (letno, "cetrtletno, mese"cno),
\item [gapfill] manjkajo"ce vrednosti prej"snjih meritev,
\item [date] datum ali obdobje,
\item [page] stran,
\item [per\_page] "stevilo elementov na stran.
\end{description}

Privzeto bo programski vmesnik vrnil podatke za vse mo"zne "casovne 
vrednosti. V odgovoru API-ja dobimo seznam objektov (Primer
\ref{dataset_response}) z datumom, indikatorjem, dr"zavo in vrednostjo.


\begin{snippet}
\begin{center}
\begin{lstlisting}
{
    "indicator": {
        "id": "SP.POP.TOTL",
        "value": "Population, total"
    },
    "country": {
        "id": "IL",
        "value": "Israel"
    },
    "value": "6289000",
    "decimal": "0",
    "date": "2000"
}
\end{lstlisting}
\end{center}
\caption{Podatki za indikator SP.POP.TOTL (populacija dr"zave) za Izrael leta
2000.}
\label{dataset_response}
\end{snippet} 


\section{Podatki podnebnih meritev}

Programski vmesnik Svetovne banke za podnebne podatke omogo"ca dostop do 
podatkov napovednih modelov in zgodovinskih meritev meteorolo"skih postaj. V tej 
diplomski nalogi smo se odlo"cili uporabiti samo podatke zgodovinskih meritev, 
saj si s temi podatki lahko uporabnik programa Orange sam sestavi svoje 
napovedne modele.

Za razliko od uporabe programskega vmesnika indikatorjev, lahko pri tem
programskem vmesniku uporabljamo veljavne ISO 3166-1 alpha-2 ali ISO 3166-1 
alpha-3 kode dr"zav, ali pa "stevilski identifikator (TODO link) vodoto"cnega 
obmo"cja.

Ta programski vmesnik nam omogo"ca dostop do podatkov o povpre"cnih temperaturah 
in padavinah v "casovnih obdobjih enega leta, desetletja ali pa nam omogo"ca 
dostop do mese"cnih povpre"cij skozi vsa leta meritev.


\subsection{Dostop do podatkov podnebnih meritev}

Dostop do podatkov podnebnih meritev je mogo"c na naslovu \\
% TODO
http://climatedataapi.worldbank.org/climateweb/rest/\\
v1/\textbf{loc\_type}/cru/\textbf{data\_type}/\textbf{interval}/\textbf{location}\\
kjer je:
\begin{description}
\item [loc\_type] vrsta identifikatorja obmo"cja (``basin'' za vodotocno obmo"cje, 
  ``country'' za dr"zave),
\item [data\_type] vrsta meritev (``pr'' za padavine, ``tas'' za temperature),
\item [interval] vrsta meritvenega obdobja (``month'' za mese"cno, ``year'' za letno in
  ``decade'' za desetletno),
\item [location] identifikator dr"zave ali vodoto"cnega obmo"cja.
\end{description}


Izsek podatkov (primer za mese"cno povpre"cje koli"cine padavin za Slovenijo):
http://climatedataapi.worldbank.org/climateweb/rest/v1/country/cru/pr/month/SVN

\begin{lstlisting}
{
    "month": 0,
    "data": 68.93643
},
{
    "month": 1,
    "data": 64.23069
},
{
    "month": 2,
    "data": 81.098724
},
...
\end{lstlisting}











\section{Te"zave pri dostopu}


Tezave pri uporabi SB API-ja lahko razdelimo v dve skupini. Prvo skupino
sestavljajo te"zave pri dokumentaciji in z manjkajo"cimi podatki, drugo pa
napake v samih pridobljenih podatkih.

Te"zave prve skupine:

\begin{itemize}  
\item ob posodabljanju spletne strani SB se izgubijo posamezne povezave do 
  primerov, dokumentacije in opisov API-ja,
\item nepopolna dokumentacija:
\subitem polje za datum je opisano, vendar ni dokumentirano, kak"sne so vse mo"zne 
    vrednosti,
\subitem delovanje polj za obdobje (date), zadnje vrednosti (mrv) in za mankjajo"ce
    vrednosti (gapfill) ni ustrezno opisano.
\end{itemize}  

Te"zave druge skupine:


\begin{itemize}  
\item manjkajo"ci identifikatoriji za polja na posameznih indikatorjih (primer je
  manjkajo"ca vrednost v polju id dr"zave),
\item datum vsebuje naklju"cne vrednosti (``last known value'' ``2001 - 2015'' ``2040''
\item zgornja meja "stevila izbranih lokacij na 250 ni navedena,
\item nemogo"ce je ugotoviti pogostost vzor"cenja indikatorja (frequency).
\end{itemize}  







