\chapter{Spletni viri indikatorjev držav sveta}

Na spletu je mnogo prosto dostopnih virov podatkov. Te vmesniki ponujajo dostop
do zelo raznolikih podatkov, kot so seznami stopnje ogrozenosti zivali po
drzavah (1), Nasini podatki meritev in slike vesolja (2), seznam knjig z 
ocenami in povezavami med uporabniki (3), zgodovina meteorolo"skih meritev (4),
razni indikatorji stopnje razvoja drzav (5).



% 1 IUCN 2016. IUCN Red List of Threatened Species. Version 2016-1 <www.iucnredlist.org>
% http://apiv3.iucnredlist.org/api/v3/docs
% 
% 2 https://api.nasa.gov/#getting-started
% 
% 3 https://www.goodreads.com/api
% 
% 4 http://climatedataapi.worldbank.org/
% 
% 5 http://api.worldbank.org/




Tukaj se bomo osredoto"cili
na dva programska vmesnika za dostop podatkov Svetovne banke. Za Svetovno banko
smo se odlo"cili, ker zdru"zuje in na enovit na"cin predstavi podatke z ve"c
razli"cnih virov. Prvi programski vmesnik je za indikatorje razvoja dr"zav,
drugi pa za dostop do vremenskih meritev po dr"zavah. Dostop do podatkov je
omogo"cen preko REST vmesnika, ki ponuja veliko moznosti za iskanje in
filtriranje rezultatov.


\section{Indikatorji razvoja dr"zav}

Programski vmesnik indikatorjev razvoja dr"zav Svetovne banke omogo"ca dostop
do preko 18000 raznih indikatorjev. Vsi podatki indikatorjev so merjeni v
obdobju od 1940 do trenutnega leta. 

Podatke iz naslednjih virov:
World Development Indicators, 
Global Development Finance, 
African development Indicators, 
Doing Business,
Enterprise Surveys, 
Millennium Development Goals, 
Education Statistics, 
Gender Statistics,
Health and Nutrition Statistics, 
IDA Results Measurement System.


\subsection{Opis programkega vmesnika indikatorjev}


Programsk vmesik za podatke o indikatorjih razvoja omogo"ca 


\subsection{Dostop do podatkov indikatorjev}

Dostop do podatkov je omogo"cen preko dostopne to"cke

http://api.worldbank.org

Tukaj najdemo vmesnike za iskanje indikatorjev (indicators), dr"zav in regij
(countries), raven dohodka (Income level), vrsta posojila (lending type), tem
(topic), virov (sources).





\section{Tezave pri dostopu}

- posodobitve spletne strani ter podatkovnega vmesnika, 404 vecina strani.

- Pomankljiva dokumentacija
    - polje za datum je opisano vendar ni dokumentirano kaksne so vse mozne
      vrednosti. 

- manjkajoci identifikatoriji za polja na nakljucnih indikatorjih.
  primer (drzava ima le ime in prazno id polje)
- datum vsebuje naklju"cne vrednosti ("last known value" "2001 - 2015", "2040")
- nedokumentira meja stevila izbranih lokacij 250
- slabo definirano delovanje filtrov podatkov (mrv in date in fill)
- Nemogoce ugotoviti ali ima indikator letne, cetrtlente ali mesecne vrednosti.



CCCCC
