\chapter{Spletni viri indikatorjev držav sveta}

Na spletu je mnogo prosto dostopnih virov podatkov. Tukaj se bomo osredoto"cili
na dva programska vmesnika za dostop podatkov Svetovne banke. Prvi programski
vmesnik je za indikatorje razvoja dr"zav, drugi pa za dostop do vremenskih
meritev po dr"zavah. Dostop do podatkov je omogo"cen preko REST vmesnika, ki
ponuja veliko moznosti za iskanje in filtriranje rezultatov.


\section{Indikatorji razvoja dr"zav}

Programski vmesnik indikatorjev razvoja drzav Svetovne banke omogo"ca dostop
do preko 18000 raznih indikatorjev. Vsi podatki indikatorjev so merjeni v
obdobju od leta 1940 do trenutnega leta. 

Podatke iz naslednjih virov:
World Development Indicators, 
Global Development Finance, 
African development Indicators, 
Doing Business,
Enterprise Surveys, 
Millennium Development Goals, 
Education Statistics, 
Gender Statistics,
Health and Nutrition Statistics, 
IDA Results Measurement System.


\subsection{Opis programkega vmesnika indikatorjev}


Programsk vmesik za podatke o indikatorjih razvoja omogo"ca 


\subsection{Dostop do podatkov indikatorjev}

Dostop do podatkov je omogo"cen preko dostopne to"cke

http://api.worldbank.org

Tukaj najdemo vmesnike za iskanje indikatorjev (indicators), dr"zav in regij
(countries), raven dohodka (Income level), vrsta posojila (lending type), tem
(topic), virov (sources).





\section{Tezave pri dostopu}

- posodobitve spletne strani ter podatkovnega vmesnika, 404 vecina strani.

- Pomankljiva dokumentacija
    - polje za datum je opisano vendar ni dokumentirano kaksne so vse mozne
      vrednosti. 

CCCCC
