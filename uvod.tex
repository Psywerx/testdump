
\chapter{Uvod}

Na svetovnem spletu je dosegljivih vedno vec prosto dostopnih programskih
vmesnikov (API, ang. application programming interface). Ti vmesniki omogocajo dostop
do zelo raznolikih baz podatkov. Nekateri primeri baz so                          
seznami stopnje ogrozenosti zivali po drzavah \fnurl{http://apiv3.iucnredlist.org/api/v3/docs},    
nasini podatki meritev in slike vesolja  \fnurl{https://api.nasa.gov/},
seznam knjig z ocenami in povezavami med uporabniki \fnurl{https://www.goodreads.com/api},
zgodovina meteoroloskih meritev, \fnurl{http://climatedataapi.worldbank.org/},
razni indikatorji stopenj razvoja drzav \fnurl{http://api.worldbank.org/}.

Programski vmesniki so oblikovani tako, da je omogocena raznolika uporaba 
podatkov iz podatkovnh baz. To pa ima tudi slabost, ki je v tem, da je podatke 
potrebno predhodno obdelati za vsak namen posebej. Tako bi na primer moral vsak
uporabnik programa Orange podatke predhodno pretvoriti v obliko, primerno za 
njegovo konkretno analizo.


\section{Motivacija}

Povezava programskega vmesnika in orodja za analizo podatkov je pogosto    
prezapletena za navadnega uporabnika. Z dodatkom ODS zelimo podatke
programske vmesnika Svetovne banke spraviti v obliko, primerno za nadaljno 
uporabo v orodju Orange. Ta dodatek bi pomagal zdruziti programe za obdelavo 
podatkov in prosto dostopnih baz podatkov. S tem dobimo enostavnejsi dostop do
podatkov iz prek 16.000 indikatorjev in stevilnik klimatskih meritev, 
s cimer bomo lazje analizirali in iskali morebitne zakonitosti med podatki.
V kolikor bi imeli en sam ustrezen dodatek (add-on) za dostop do podatkov 
programskega vmesnika Svetovne banke, bi poenostavili posodabljanje in 
vzdrzevanje kode v primeru sprememb programskega vmesnika za vse uporabnike
istega orodja hkrati. S tem odpravimo potrebo, da bi moral vsak uporabnik sam
skrbeti za uskladitvene posodobitve.


\section{Cilji in struktura diplomske naloge}

Cilj diplomske naloge je izdelati knjiznico za uporabo programskega vmesnika
Svetovne banke ter izdelati dodatek za program Orange, ki s pomocjo omenjene
knjiznice omogoca uporabniku dostop do podatkov SB preko graficnega vmesnika.


V drugem poglavju diplomskega dela predstavimo spletne vire indikatorjev
drzav sveta. Nato bomo podrobneje opisali programski vmesnik za dostop do 
podatkov Svetovne banke (API SB). V cetrtem poglavju sledi predstavitev 
knjiznice in gradnikov za Orange in nato se konkretni primeri uporabe. Na koncu
bomo popisali opravljeno delo, navedli vire kode in opisali nadaljne moznosti
nadgradnje dodatka.










                                                                                
                                                                                
                         

-------------------------------

 - Prav tako se ve"cina programov in knji"znic za dostop do baz podatkov osredoto"ci le na iskanje po teh bazah, ne pa tudi na pridobivanje "cim ve"cje koli"cine podatkov.                                    
                                                                                

% Poleg tega obstaja dosti odprto kodnih programov za obdelavo                  
% in analizo podatkov. Ker so programski vmesnike bolj splo"sno namenski, je       
                                                                                

%% \chapter{Uvod}
%% 
%% 
%% Na spletu je vedno ve"c prosto dostopnih programskih vmesnikov(API, ang. 
%% application programming interface) za razli"cne baze podatkov. Ti vmesniki 
%% ponujajo dostop do zelo raznolikih podatkov, kot so 
%% seznami stopnje ogro"zenosti "zivali po dr"zavah 
%% \fnurl{http://apiv3.iucnredlist.org/api/v3/docs}, 
%% NASA podatki meritev in slike vesolja 
%% \fnurl{https://api.nasa.gov/}, 
%% seznam knjig z ocenami in povezavami med uporabniki 
%% \fnurl{https://www.goodreads.com/api},
%% zgodovina meteorolo"skih meritev
%% \fnurl{http://climatedataapi.worldbank.org/}, 
%% razni indikatorji stopnje razvoja dr"zav
%% \fnurl{http://api.worldbank.org/}.
%% 
%% Ker pa so programski vmesniki bolj splo"sno namenski, je
%% podatke te"sko spraviti v obliko ki bi bila primerna za uporabo v raznih 
%% orodjih za analizo in obdelavo podatkov. Prav tako se ve"cina programov in knji"znic za dostop
%% do baz podatkov osredoto"ci le na iskanje po teh bazah, ne pa tudi na
%% pridobivanje "cim ve"cje koli"cine podatkov.
%% 
%% 
%% % Poleg tega obstaja dosti odprto kodnih programov za obdelavo 
%% % in analizo podatkov. Ker so programski vmesnike bolj splo"sno namenski, je
%% % podatke te"sko spraviti v obliko za analizo in obdelavo. Z orodjem, ki bi
%% % pomagalo zdru"ziti programe za obdelavo podatkov in prosto dostopne baze
%% % podatkov, bi omogo"cili raziskovanje teh podatkov "sir"si javnosti.
%% 
%% 
%% 
%% % Povezava programskega vmesnika in orodja za analizo podatkov pa je pogosto 
%% % prezapletena za navadnega uporabnika. Z dodajanjem gradnikov za enostavno 
%% % uporabo spletnih programskih vmesnikov v orodjih kot je Orange, omogo"cimo ...
%% 
%% \section{Motivacija}
%% 
%% Branje podatkov z raznih programskih vmesnikov je lahko zelo zamudno delo.
%% Programski vmesnik se lahko s "casom spremeni, in podatki ki jih dobimo z
%% vmesnika so lahko pokvarjeni. Trenutni pristop, kjer moramo podatke vsaki"c ro"cno
%% obilkovati da so primerni za analizo, ima mnogo pomanjkljivosti. Prejeti podatki
%% lahko vsebujejo nepravilnosti, ali pa so celo nedostopni. Z dodatkom ki bi
%% poskrbel za prenos podatkov in pretovrbo v uporabno obliko, hkrati pa bi znal
%% popraviti ali odstraniti pokvarjene podatke, bi lahko ve"c pozornosti posvetili
%% sami analizi in obdelavi. Poleg tega, pa ve"cina knji"znic za delo z odprtimi
%% programskimi vmesniki, nudi zelo dobre na"cine za iskanje posameznih podatkov,
%% ne pa za prenos ve"cje koli"cine podatkov kar je bolj primerno za analizo in
%% obdelavo.
%% 
%% 
%% 
%% \section{Cilji}
%% 
%% Cilj diplomske naloge je izdelati knji"znico, ki omogo"ca enostaven dostop do
%% podatkov Svetovne banke in interaktivni gradnik v programu Orange za dostop in
%% uporabo teh podatkov. S tem bomo omogo"cili raziskovanje teh podtakov "sir"si
%% javnosti. Knji"znica bo poenostavila prenos ve"cjega "stevila podatkov, in
%% predstavila te podatke v obliki primerni za orodje Orange. k<
