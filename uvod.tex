
\chapter{Uvod}

Na svetovnem spletu je dosegljivih vedno več prosto dostopnih programskih
vmesnikov \angl{application programming interface}. 
% TODO: morda dodaj API v razlago zgoraj
Ti vmesniki omogočajo dostop
do zelo raznolikih zbirk podatkov. Nekaj primerov prosto dostopnih podatkovnih
zbirk je seznam stopnje ogroženosti živali po državah 
% 1 IUCN 2016. IUCN Red List of Threatened Species. Version 2016-1 <www.iucnredlist.org>
    \fnurl{http://apiv3.iucnredlist.org/api/v3/docs},
podatki meritev in slike vesolja agencije NASA
    \fnurl{https://api.nasa.gov/},
seznam knjig z ocenami in povezavami med uporabniki 
    \fnurl{https://www.goodreads.com/api},
zgodovina meteoroloških meritev 
    \fnurl{http://climatedataapi.worldbank.org/},
razni indikatorji stopenj razvoja držav
    \fnurl{http://api.worldbank.org/}.

Programski vmesniki so oblikovani tako, da je omogočena raznolika uporaba
podatkov iz podatkovnih zbirk. To pa ima tudi slabost, ki je v tem, da je 
podatke potrebno predhodno obdelati za vsak namen posebej. Tako bi na primer 
moral vsak uporabnik programa Orange podatke predhodno pretvoriti v obliko, 
primerno za njegovo konkretno analizo.


\section{Motivacija}

% TODO: tri motivacije: 
% lazja uporaba laikom - gui klikanje,
% lazje vzrdzevanje in posodobobitve, en codebase,
% ne rabi usak delat iste stvari, rocno gradit tab separate fajle, manj napak,
%   usak skrbeti za iste pomankljivosti apija (slabi podatki itd).

Povezava programskega vmesnika za dostop do podatkov in orodja za analizo 
podatkov je pogosto prezapletena za navadnega uporabnika. Z dodatkom Orange
data sets želimo podatke programskega vmesnika Svetovne banke spraviti v 
obliko, primerno za nadaljnjo
uporabo v orodju Orange. Ta dodatek bi pomagal združiti programe za obdelavo
podatkov in prosto dostopne zbirke podatkov. S tem dobimo enostavnejši dostop 
do podatkov iz prek 16.000 indikatorjev in številnih podnebnih meritev,
s čimer bomo lažje analizirali in iskali morebitne zakonitosti v podatkih.
Če bi imeli en sam ustrezen dodatek za dostop do podatkov programskega 
vmesnika Svetovne banke, bi poenostavili posodabljanje in
vzdrževanje kode v primeru sprememb programskega vmesnika za vse uporabnike
istega orodja hkrati. S tem odpravimo potrebo, da bi moral vsak uporabnik sam
skrbeti za uskladitvene posodobitve.


\section{Cilji in struktura diplomske naloge}

Cilj diplomske naloge je izdelati knjižnico za uporabo programskega vmesnika
Svetovne banke ter izdelati dodatek za program Orange, ki s pomočjo omenjene
knjižnice omogoča uporabniku dostop do podatkov Svetovne banke preko
grafičnega vmesnika.

V diplomski nalogi najprej predstavimo spletna vira indikatorjev
držav sveta in meritev podnebnih podatkov Svetovne banke, ter
opišemo delovanje njunih programskih vmesnikov.
Nato podrobneje opišemo našo implementacijo knjižnice za dostop do
programskega vmesnika Svetovne banke in gradnikov za program Orange, ki to
knjižnico uporabljajo. V nadaljevanju prikažemo še nekaj praktičnih 
primerov uporabe dodatka Orange data sets. Na koncu še popišemo opravljeno 
delo, navedemo vire kode in omenimo možne načine za izboljšavo ali 
nadgradnjo našega dodatka.












% bomo podrobneje opisali programski vmesnik za dostop do
% podatkov Svetovne banke (API SB). V četrtem poglavju sledi predstavitev
% knjižnice in gradnikov za Orange in nato še konkretni primeri uporabe. Na koncu
% bomo popisali opravljeno delo, navedli vire kode in opisali nadaljne možnosti
% nadgradnje dodatka.


%% -------------------------------
%%
%%  - Prav tako se večina programov in knjižnic za dostop do baz podatkov osredotoči le na iskanje po teh bazah, ne pa tudi na pridobivanje čim večje količine podatkov.
%%
%%
%% % Poleg tega obstaja dosti odprto kodnih programov za obdelavo
%% % in analizo podatkov. Ker so programski vmesnike bolj splošno namenski, je
%%
%%
%% \chapter{Uvod}
%%
%%
%% Na spletu je vedno več prosto dostopnih programskih vmesnikov(API, ang.
%% application programming interface) za različne baze podatkov. Ti vmesniki
%% ponujajo dostop do zelo raznolikih podatkov, kot so
%% seznami stopnje ogroženosti živali po državah
%% \fnurl{http://apiv3.iucnredlist.org/api/v3/docs},
%% NASA podatki meritev in slike vesolja
%% \fnurl{https://api.nasa.gov/},
%% seznam knjig z ocenami in povezavami med uporabniki
%% \fnurl{https://www.goodreads.com/api},
%% zgodovina meteoroloških meritev
%% \fnurl{http://climatedataapi.worldbank.org/},
%% razni indikatorji stopnje razvoja držav
%% \fnurl{http://api.worldbank.org/}.
%%
%% Ker pa so programski vmesniki bolj splošno namenski, je
%% podatke teško spraviti v obliko ki bi bila primerna za uporabo v raznih
%% orodjih za analizo in obdelavo podatkov. Prav tako se večina programov in knjižnic za dostop
%% do baz podatkov osredotoči le na iskanje po teh bazah, ne pa tudi na
%% pridobivanje čim večje količine podatkov.
%%
%%
%% % Poleg tega obstaja dosti odprto kodnih programov za obdelavo
%% % in analizo podatkov. Ker so programski vmesnike bolj splošno namenski, je
%% % podatke teško spraviti v obliko za analizo in obdelavo. Z orodjem, ki bi
%% % pomagalo združiti programe za obdelavo podatkov in prosto dostopne baze
%% % podatkov, bi omogočili raziskovanje teh podatkov širši javnosti.
%%
%%
%%
%% % Povezava programskega vmesnika in orodja za analizo podatkov pa je pogosto
%% % prezapletena za navadnega uporabnika. Z dodajanjem gradnikov za enostavno
%% % uporabo spletnih programskih vmesnikov v orodjih kot je Orange, omogočimo ...
%%
%% \section{Motivacija}
%%
%% Branje podatkov z raznih programskih vmesnikov je lahko zelo zamudno delo.
%% Programski vmesnik se lahko s časom spremeni, in podatki ki jih dobimo z
%% vmesnika so lahko pokvarjeni. Trenutni pristop, kjer moramo podatke vsakič ročno
%% obilkovati da so primerni za analizo, ima mnogo pomanjkljivosti. Prejeti podatki
%% lahko vsebujejo nepravilnosti, ali pa so celo nedostopni. Z dodatkom ki bi
%% poskrbel za prenos podatkov in pretovrbo v uporabno obliko, hkrati pa bi znal
%% popraviti ali odstraniti pokvarjene podatke, bi lahko več pozornosti posvetili
%% sami analizi in obdelavi. Poleg tega, pa večina knjižnic za delo z odprtimi
%% programskimi vmesniki, nudi zelo dobre načine za iskanje posameznih podatkov,
%% ne pa za prenos večje količine podatkov kar je bolj primerno za analizo in
%% obdelavo.
%%
%%
%%
%% \section{Cilji}
%%
%% Cilj diplomske naloge je izdelati knjižnico, ki omogoča enostaven dostop do
%% podatkov Svetovne banke in interaktivni gradnik v programu Orange za dostop in
%% uporabo teh podatkov. S tem bomo omogočili raziskovanje teh podtakov širši
%% javnosti. Knjižnica bo poenostavila prenos večjega števila podatkov, in
%% predstavila te podatke v obliki primerni za orodje Orange. k<
